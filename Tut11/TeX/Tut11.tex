%% LaTeX-Beamer template for KIT design
%% by Erik Burger, Christian Hammer
%% title picture by Klaus Krogmann
%%
%% version 2.1
%%
%% mostly compatible to KIT corporate design v2.0
%% http://intranet.kit.edu/gestaltungsrichtlinien.php
%%
%% Problems, bugs and comments to
%% burger@kit.edu

\documentclass[18pt]{beamer}

%% SLIDE FORMAT

% use 'beamerthemekit' for standard 4:3 ratio
% for widescreen slides (16:9), use 'beamerthemekitwide'

\usepackage{templates/beamerthemekit}
% \usepackage{templates/beamerthemekitwide}

%% TITLE PICTURE

% if a custom picture is to be used on the title page, copy it into the 'logos'
% directory, in the line below, replace 'mypicture' with the 
% filename (without extension) and uncomment the following line
% (picture proportions: 63 : 20 for standard, 169 : 40 for wide
% *.eps format if you use latex+dvips+ps2pdf, 
% *.jpg/*.png/*.pdf if you use pdflatex)

%\titleimage{mypicture}

%% TITLE LOGO

% for a custom logo on the front page, copy your file into the 'logos'
% directory, insert the filename in the line below and uncomment it

%\titlelogo{mylogo}

% (*.eps format if you use latex+dvips+ps2pdf,
% *.jpg/*.png/*.pdf if you use pdflatex)

%% TikZ INTEGRATION

% use these packages for PCM symbols and UML classes
% \usepackage{templates/tikzkit}
% \usepackage{templates/tikzuml}

% the presentation starts here
\usepackage{graphicx}
\usepackage{listings}
\usepackage{color}
\usepackage{textcomp}
\definecolor{listinggray}{gray}{0.9}
\definecolor{lbcolor}{rgb}{0.9,0.9,0.9}
\lstset{
	language=Java,
	backgroundcolor=\color{lbcolor},
	tabsize=4,
	rulecolor=,
        basicstyle=\footnotesize,
        aboveskip=5pt,
        upquote=true,
        columns=fixed,
        showstringspaces=false,
        extendedchars=true,
        breaklines=true,
        frame=single,
        showtabs=false,
        showspaces=false,
        showstringspaces=false,
        identifierstyle=\ttfamily,
        keywordstyle=\color[rgb]{0,0,1},
        commentstyle=\color[rgb]{0.133,0.545,0.133},
        stringstyle=\color[rgb]{0.627,0.126,0.941},
}
%\setbeamercovered{covered}

\usepackage[utf8]{inputenc}
\usepackage{stmaryrd}
\setbeamercovered{covered}

\title[Prog Tut Nr. 11]{Tutorium Programmieren}
\subtitle{Tut Nr.11: Exceptions, java.util}
\author{Michael Friedrich}
\date{21. / 23.11.2013}

\institute{Institut f\"ur theoretische Informatik}
% Bibliography

\usepackage[citestyle=authoryear,bibstyle=numeric,hyperref,backend=biber]{biblatex}
\addbibresource{templates/example.bib}
\bibhang1em

\begin{document}

% change the following line to "ngerman" for German style date and logos
\selectlanguage{ngerman}

%title page
\begin{frame}
	\titlepage
\end{frame}

%table of contents
\begin{frame}{Outline/Gliederung}

	\tableofcontents
\end{frame}

\section{Exceptions}
\begin{frame}[fragile]{Exceptions}
\begin{lstlisting}
public static void main(String[] args) {
   String input = "";
   BufferedReader buf = new BufferedReader( new InputStreamReader(System.in));
   while(!input.equals("quit")) {  
     try {
       input = buf.readLine();
     } catch (IOException e1) {
       System.out.println("unable to read - shutting down...");
			 input = "quit";
     }
     try {
       int i = Integer.parseInt(input);
     } catch (NumberFormatException e) {
       System.out.println("Eingabe war keine Zahl!");
     }
   }
  System.out.println("shutting down...");
}\end{lstlisting}
\end{frame}

\begin{frame}{Was war bei dem Beispiel wichtig?!}
\pause
\begin{itemize}
	\item Programm stürzt nie unkontrolliert ab \pause
	\item User wird über seine Fehler informiert und kann diese verbessern  \pause
\end{itemize} \pause
Wir als Programmierer müssen unfähigen User (und Kollegen...) entgegen arbeiten. \newline  
Beispiel?  \pause NullPointer abfangen, falsche Werte geliefert, falsche Formattierung...

%%Bild user <--> programmers
\end{frame}

\section{java.util}
\begin{frame}[fragile]{java.util}
Java bietet von sich aus schon sehr viel Funktionalität, z.Bsp LinkedList. \pause
\begin{example}{weitere eingebaute Datenstrukturen} \pause
\begin{itemize}
	\item Collections: ungeordneter Pool an Objekten 
	\begin{itemize}
		\item \lstinline{Collection<Product> products;} \pause
	\end{itemize}
	\item SortedSet: Menge mit totaler Ordnung
	\begin{itemize}
		\item \lstinline{SortedSet<Product> products;} \pause
		\item Product MUSS hier Comparable$<$Product$>$ implementieren \pause
	\end{itemize}
	\item ArrayList: ähnlich LinkedList, aber mit Index
	\begin{itemize}
		\item \lstinline{ArrayList<Product> products;} \pause
	\end{itemize}
	\item Maps: key-value Paare 
	\begin{itemize}
		\item \lstinline{TreeMap<Product, Customer> orders;}
		\item \lstinline{HashMap<Product, Customer> orders;}
	\end{itemize}
\end{itemize}
\end{example}
Nehmt die Hinweise auf dem Übungsblatt als Einstiegspunkt.
\end{frame}


\appendix
\beginbackup

%\begin{frame}[allowframebreaks]{References}
%	\printbibliography
%\end{frame}

\backupend

\end{document}
