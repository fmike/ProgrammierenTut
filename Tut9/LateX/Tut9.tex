%% LaTeX-Beamer template for KIT design
%% by Erik Burger, Christian Hammer
%% title picture by Klaus Krogmann
%%
%% version 2.1
%%
%% mostly compatible to KIT corporate design v2.0
%% http://intranet.kit.edu/gestaltungsrichtlinien.php
%%
%% Problems, bugs and comments to
%% burger@kit.edu

\documentclass[18pt]{beamer}

%% SLIDE FORMAT

% use 'beamerthemekit' for standard 4:3 ratio
% for widescreen slides (16:9), use 'beamerthemekitwide'

\usepackage{templates/beamerthemekit}
% \usepackage{templates/beamerthemekitwide}

%% TITLE PICTURE

% if a custom picture is to be used on the title page, copy it into the 'logos'
% directory, in the line below, replace 'mypicture' with the 
% filename (without extension) and uncomment the following line
% (picture proportions: 63 : 20 for standard, 169 : 40 for wide
% *.eps format if you use latex+dvips+ps2pdf, 
% *.jpg/*.png/*.pdf if you use pdflatex)

%\titleimage{mypicture}

%% TITLE LOGO

% for a custom logo on the front page, copy your file into the 'logos'
% directory, insert the filename in the line below and uncomment it

%\titlelogo{mylogo}

% (*.eps format if you use latex+dvips+ps2pdf,
% *.jpg/*.png/*.pdf if you use pdflatex)

%% TikZ INTEGRATION

% use these packages for PCM symbols and UML classes
% \usepackage{templates/tikzkit}
% \usepackage{templates/tikzuml}

% the presentation starts here
\usepackage{graphicx}
\usepackage{listings}
\usepackage{color}
\usepackage{textcomp}
\definecolor{listinggray}{gray}{0.9}
\definecolor{lbcolor}{rgb}{0.9,0.9,0.9}
\lstset{
	language=Java,
	backgroundcolor=\color{lbcolor},
	tabsize=2,
	rulecolor=,
        basicstyle=\footnotesize,
        aboveskip=5pt,
        upquote=true,
        columns=fixed,
        showstringspaces=false,
        extendedchars=true,
        breaklines=true,
        frame=single,
        showtabs=false,
        showspaces=false,
        showstringspaces=false,
        identifierstyle=\ttfamily,
        keywordstyle=\color[rgb]{0,0,1},
        commentstyle=\color[rgb]{0.133,0.545,0.133},
        stringstyle=\color[rgb]{0.627,0.126,0.941},
			}
\setbeamercovered{covered}
\usepackage[utf8]{inputenc}
\usepackage{stmaryrd}

\title[Prog Tut Nr. 9]{Tutorium Programmieren}
\subtitle{Tut Nr.9: Vererbung}
\author{Michael Friedrich}
\date{07. / 09.01.2014}

\institute{Institut f\"ur theoretische Informatik}
% Bibliography

\usepackage[citestyle=authoryear,bibstyle=numeric,hyperref,backend=biber]{biblatex}
\addbibresource{templates/example.bib}
\bibhang1em

\begin{document}

% change the following line to "ngerman" for German style date and logos
\selectlanguage{ngerman}

%title page
\begin{frame}
	\titlepage
\end{frame}

%table of contents
\begin{frame}{Outline/Gliederung}

	\tableofcontents
\end{frame}

\section{Übungsblatt4}
\begin{frame}{Übungsblatt4}
\begin{itemize}
	\item Nutzt die öffentlichen Test als Hilfe \newline \pause
	$\shortrightarrow$ Bei Abschlussaufgaben Abgabekriterium! \newline
	ABER kein hardcoden!!! - Zählt als Betrugsversuch
\end{itemize}
\end{frame}

\section{Vererbung}
\begin{frame}[fragile]{Einführung}
Wird genutzt, um Gleiches zu "`gruppieren"'
\begin{exampleblock}{Beispiel}
\begin{lstlisting}
class Person {
  String name;
  int age;
}
class Student extends Person {
	   int matriculation;
  Tutor tutor;
}
class Tutor extends Student {
	//hat auch name, age, matriculation
	String name; //anderes name als Person.name!
  Student[] students;
}
\end{lstlisting}
\end{exampleblock}
\color[rgb]{1,0,0}{ACHTUNG} Überschreiben von Attributen nicht möglich
\end{frame}

\begin{frame}[fragile]{Aufpassen!}
\begin{itemize}
\item Vererbung wird gerne falsch gemacht! $\Rightarrow$ immer IST-Beziehung
\end{itemize}
\begin{exampleblock}{SO NICHT}
\begin{lstlisting}
class Point {
  int x, y;
}
class Linie extends Point {
  int x2, y2; // huh?
}
\end{lstlisting}
\end{exampleblock} \pause
\begin{exampleblock}{Sondern?} \pause
\begin{lstlisting}
class Linie {
  Point p1, p2; //HAT-Beziehung
}
\end{lstlisting}
\end{exampleblock}
\end{frame}

\begin{frame}[fragile]{Vererbung von Methoden}
\begin{itemize}
	\item habt ihr alle schon implizit genutzt, Beispiele? \pause
	\begin{itemize}
		\item \lstinline{toString, equals(Object o), ...} \pause
	\end{itemize}
	\item diese Methoden erbt JEDE Klasse von \lstinline{java.lang.Object} \pause \newline
	Habt ihr schon überschrieben (daher @Override) \pause
	\item Konstruktoren werden mitvererbt und \textbf{müssen immer} in der Subklasse implementiert werden \newline
	$\shortrightarrow$ Zugriff auf den Oberklassen Methoden (also auch Konstruktor) über \lstinline{super()} \pause
\end{itemize}
\begin{exampleblock}{super}
\begin{lstlisting}
public Student(name, age, matriculation) {
 super(name, age);
	 this.matriculation = matriculatio;
}
\end{lstlisting}
\end{exampleblock}
\end{frame}

\begin{frame}[fragile]{Polymorphismus}
\begin{itemize}
	\item Eine Unterklasse kann alles, was die Oberklasse kann \newline
	$\shortrightarrow$ exakt gleiche Methoden, auch wenn eventuell überschrieben \newline 
\end{itemize}
Beispiel: 
\begin{lstlisting}[mathescape]
	Student michael = new Student();
	Person mike = michael; //Jeder Student IST also gleichzeitig eine Person 
	Person anna = new Student();
	Student anne = new Person(); $ \color[rgb]{1,0,0}{{\lightning} NEIN!}$
\end{lstlisting}
\end{frame}

\begin{frame}[fragile]{Abstrakte (\lstinline{abstract}) Klassen}
\begin{itemize}
	\item Klassen: Nicht instantiierbar
	\item Methoden: MÜSSEN in Subklasse instantiiert werden
	\begin{itemize}
		\item nur in abstrakten Klassen erlaubt \pause
	\end{itemize}
\end{itemize}
\begin{lstlisting}
	abstract class mammal {
	//protected nur in erbenden Klassen sichtbar
  protected String s = "schnaufen";

  abstract public void breathe();
}

class human extends mammal {

  public void breathe() {
    System.out.print(s);
  }
}
\end{lstlisting}
\end{frame}

\begin{frame}[fragile]{Interfaces - \lstinline{implements}}
\begin{itemize}
	\item ähnlich den abstrakten Klassen aber keine Attribute \pause
	\item definieren Schnittstelle  
	\begin{itemize}
		\item also anderer Sinn als abstrakte Klassen!  \pause
		\item definieren die Methoden, die eine Klasse implementieren \textbf{MUSS}
	\end{itemize}
\end{itemize}
\begin{lstlisting}[mathescape]
public interface Comparable {
  boolean equals();
  int compareTo();
}
public Letter implements Comparable {
  char c;
	
  public boolean equals(Letter l) {
    return this.c==l;
  } 
	public int compareTo(Letter l) {
	  //zBsp alphabetisch steigend
	}
}
\end{lstlisting} \pause
Methoden automatisch \lstinline{abstract} und \lstinline{public}
\end{frame}

\begin{frame}[fragile]{Interfaces}
\begin{itemize}
	\item Eine Klasse kann beliebig viele Interfaces implementieren \pause
	\item Beziehung zwischen Interfaces über \lstinline{extends} möglich \pause
\end{itemize}
\end{frame}

\begin{frame}[fragile]{Aufgabe}
Tut Aufgabe zu Vererbung/Polymorphismus
\end{frame}


\section{Rekursion}
\begin{frame}[fragile]{Rekursion}
Bedeutung: Selbstaufruf einer Funktion \pause
\begin{lstlisting}
public class Fibs {
 public static int fib(final int n) {
    if (n <= 1) {
      return 1;
    }
    return fib(n - 1) + fib(n - 2);
  }
}
\end{lstlisting} \pause
\begin{itemize}
	\item Was passiert bei dem Aufruf von \lstinline{fib(4)}? \pause
	\item Man braucht immer einen \textbf{terminierenden} Fall! 
\end{itemize}
\end{frame}

\section{Generics}
\begin{frame}[fragile]{Generics}
Wie schon letztes Blatt gemerkt, ein Tool, um Gleiches zu komprimieren
$\shortrightarrow$ Listen sind identisch, bis auf ihren Inhalt \pause
\begin{lstlisting}
public class SinglyLinkedList<E> {
  private Node<E> head;

  public void add(E object) {
     Node<E> newNode = new Node<E>(object);
     newNode.setNext(head);
     head = newNode;
  }
}
\end{lstlisting} \pause
"`E"' ist dann durch euren gewünschter Datentyp ersetzbar, zB \lstinline{new SinglyLinkedList<Article>} \newline \pause
Aufgabe: restliche Liste mit Generics implementieren
\end{frame}


\appendix
\beginbackup

%\begin{frame}[allowframebreaks]{References}
%	\printbibliography
%\end{frame}

\backupend

\end{document}
