%% LaTeX-Beamer template for KIT design
%% by Erik Burger, Christian Hammer
%% title picture by Klaus Krogmann
%%
%% version 2.1
%%
%% mostly compatible to KIT corporate design v2.0
%% http://intranet.kit.edu/gestaltungsrichtlinien.php
%%
%% Problems, bugs and comments to
%% burger@kit.edu

\documentclass[18pt]{beamer}

%% SLIDE FORMAT

% use 'beamerthemekit' for standard 4:3 ratio
% for widescreen slides (16:9), use 'beamerthemekitwide'

\usepackage{templates/beamerthemekit}
% \usepackage{templates/beamerthemekitwide}

%% TITLE PICTURE

% if a custom picture is to be used on the title page, copy it into the 'logos'
% directory, in the line below, replace 'mypicture' with the 
% filename (without extension) and uncomment the following line
% (picture proportions: 63 : 20 for standard, 169 : 40 for wide
% *.eps format if you use latex+dvips+ps2pdf, 
% *.jpg/*.png/*.pdf if you use pdflatex)

%\titleimage{mypicture}

%% TITLE LOGO

% for a custom logo on the front page, copy your file into the 'logos'
% directory, insert the filename in the line below and uncomment it

%\titlelogo{mylogo}

% (*.eps format if you use latex+dvips+ps2pdf,
% *.jpg/*.png/*.pdf if you use pdflatex)

%% TikZ INTEGRATION

% use these packages for PCM symbols and UML classes
% \usepackage{templates/tikzkit}
% \usepackage{templates/tikzuml}

% the presentation starts here
\usepackage{graphicx}
\usepackage{listings}
\usepackage{color}
\usepackage{textcomp}
\definecolor{listinggray}{gray}{0.9}
\definecolor{lbcolor}{rgb}{0.9,0.9,0.9}
\lstset{
	language=Java,
	backgroundcolor=\color{lbcolor},
	tabsize=4,
	rulecolor=,
        basicstyle=\footnotesize,
        aboveskip=5pt,
        upquote=true,
        columns=fixed,
        showstringspaces=false,
        extendedchars=true,
        breaklines=true,
        frame=single,
        showtabs=false,
        showspaces=false,
        showstringspaces=false,
        identifierstyle=\ttfamily,
        keywordstyle=\color[rgb]{0,0,1},
        commentstyle=\color[rgb]{0.133,0.545,0.133},
        stringstyle=\color[rgb]{0.627,0.126,0.941},
}

\usepackage[utf8]{inputenc}

\title[Prog Tut Nr. 4]{Tutorium Programmieren}
\subtitle{Tut Nr.4: Kontrollstrukturen}
\author{Michael Friedrich}
\date{19. / 21.11.2013}

\institute{Institut f\"ur theoretische Informatik}
% Bibliography

\usepackage[citestyle=authoryear,bibstyle=numeric,hyperref,backend=biber]{biblatex}
\addbibresource{templates/example.bib}
\bibhang1em

\begin{document}

% change the following line to "ngerman" for German style date and logos
\selectlanguage{ngerman}

%title page
\begin{frame}
	\titlepage
\end{frame}

%table of contents
\begin{frame}{Outline/Gliederung}
	\tableofcontents
\end{frame}

\section{Wiederholung}

\section{Tutoriumsaufgabe}
\subsection{while-Schleifen}

\begin{frame}[fragile]{while - Schleifen}
Aufgabe 1 (Schleifen)
Schreiben Sie in einer Klasse namens Loops die Methoden
\begin{itemize}
\item public static boolean isPrimeWhile(int candidate) und
\item public static boolean isPrimeDoWhile(int candidate).
\end{itemize}
\small{Diese sollen als Rückgabewert den Wert true haben, falls candidate eine Primzahl ist. In den
Methoden soll dabei eine while-Schleife (a)) bzw. eine do-while-Schleife (b))
verwendet werden. Beachten Sie auch die Fälle, in denen der Wert des Parameters ganz offensichtlich
keine Primzahl ist.}
\begin{itemize}
\item Schreiben Sie zusätzlich eine main-Methode
public static void main(String[] args),
die alle Zahlen zwischen -1 und einer festgelegten konstanten Zahl N auf ihre Primzahl-Eigenschaft
überprüft.
\end{itemize}
\end{frame}


\appendix
\beginbackup

%\begin{frame}[allowframebreaks]{References}
%	\printbibliography
%\end{frame}

\backupend

\end{document}
