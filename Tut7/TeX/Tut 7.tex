%% LaTeX-Beamer template for KIT design
%% by Erik Burger, Christian Hammer
%% title picture by Klaus Krogmann
%%
%% version 2.1
%%
%% mostly compatible to KIT corporate design v2.0
%% http://intranet.kit.edu/gestaltungsrichtlinien.php
%%
%% Problems, bugs and comments to
%% burger@kit.edu

\documentclass[18pt]{beamer}

%% SLIDE FORMAT

% use 'beamerthemekit' for standard 4:3 ratio
% for widescreen slides (16:9), use 'beamerthemekitwide'

\usepackage{templates/beamerthemekit}
% \usepackage{templates/beamerthemekitwide}

%% TITLE PICTURE

% if a custom picture is to be used on the title page, copy it into the 'logos'
% directory, in the line below, replace 'mypicture' with the 
% filename (without extension) and uncomment the following line
% (picture proportions: 63 : 20 for standard, 169 : 40 for wide
% *.eps format if you use latex+dvips+ps2pdf, 
% *.jpg/*.png/*.pdf if you use pdflatex)

%\titleimage{mypicture}

%% TITLE LOGO

% for a custom logo on the front page, copy your file into the 'logos'
% directory, insert the filename in the line below and uncomment it

%\titlelogo{mylogo}

% (*.eps format if you use latex+dvips+ps2pdf,
% *.jpg/*.png/*.pdf if you use pdflatex)

%% TikZ INTEGRATION

% use these packages for PCM symbols and UML classes
% \usepackage{templates/tikzkit}
% \usepackage{templates/tikzuml}

% the presentation starts here
\usepackage{graphicx}
\usepackage{listings}
\usepackage{color}
\usepackage{textcomp}
\definecolor{listinggray}{gray}{0.9}
\definecolor{lbcolor}{rgb}{0.9,0.9,0.9}
\lstset{
	language=Java,
	backgroundcolor=\color{lbcolor},
	tabsize=4,
	rulecolor=,
        basicstyle=\footnotesize,
        aboveskip=5pt,
        upquote=true,
        columns=fixed,
        showstringspaces=false,
        extendedchars=true,
        breaklines=true,
        frame=single,
        showtabs=false,
        showspaces=false,
        showstringspaces=false,
        identifierstyle=\ttfamily,
        keywordstyle=\color[rgb]{0,0,1},
        commentstyle=\color[rgb]{0.133,0.545,0.133},
        stringstyle=\color[rgb]{0.627,0.126,0.941},
}
\usepackage[utf8]{inputenc}
\usepackage{stmaryrd}

\title[Prog Tut Nr. 7]{Tutorium Programmieren}
\subtitle{Tut Nr.7: Pakete, Listen}
\author{Michael Friedrich}
\date{10. / 12.12.2013}

\institute{Institut f\"ur theoretische Informatik}
% Bibliography

\usepackage[citestyle=authoryear,bibstyle=numeric,hyperref,backend=biber]{biblatex}
\addbibresource{templates/example.bib}
\bibhang1em

\begin{document}

% change the following line to "ngerman" for German style date and logos
\selectlanguage{ngerman}

%title page
\begin{frame}
	\titlepage
\end{frame}

%table of contents
\begin{frame}{Outline/Gliederung}
	\tableofcontents
\end{frame}

\section{Anmerkungen}
\begin{frame}{Anmeldung Übungsschein}
	Meldet euch möglichst bald an \newline
	\vspace{2cm}
	\huge \textbf{Spätestens bis zum \underline{23.12.2013}}
\end{frame}

\begin{frame}[fragile]{Feedback}
\begin{itemize}
	\item Vielen Dank nochmals\pause
	\vspace{1cm}
	\item Verbesserungspunkte\pause
	
	\begin{itemize}
		\item Fehler in den Folien \pause
		$\shortrightarrow$ JA, aber stören idR den Hauptaussage der Folie nicht. Wir sind auch nur Menschen ;) \pause
		\item Mehr zusammen coden \newline \pause
		$\shortrightarrow$ JAIN, Probleme dabei sind fehlende initiative \newline \pause
		$\shortrightarrow$ Jede Woche auseinandersetzen mit Umsetzungsschwierigkeiten \newline \pause
		$\shortrightarrow$ step-by-step durch die Lösung meiner Meinung nach besser als dauernd zuhören (Bsp. letztes Tut)
	\end{itemize}
\end{itemize}
\end{frame}


\section{Pakete}
\begin{frame}[fragile]{Pakete} \pause
\begin{itemize}
	\item Struktur, um Code zu ordnen \pause
	\item bekanntes Beispiel: Libraries \pause
\end{itemize}
\begin{exampleblock}{Beispiel}
\begin{lstlisting}[basicstyle=\scriptsize]
package garage;

public class Bike {
...
}

\end{lstlisting}
\end{exampleblock} \pause
Wird in der Regel auf dem Übungsblatt vorgegeben.
\end{frame}

\section{Listen}
\begin{frame}[fragile]{Listen} \pause
\begin{itemize}
	\item Basieren auf der Objektreferenz \pause
	\item besteht aus Nodes mit Wert Referenz auf nächsten Node \pause
\end{itemize}
\begin{exampleblock}{Beispiel}
\begin{lstlisting}[basicstyle=\scriptsize]
class ListNode{
	String content;
	ListNode next;
	
	ListNode(String x, ListNode n) {
	this.content = x;
	this.next = n;
	}
}
\end{lstlisting}
\end{exampleblock} \pause
\begin{figure}
\includegraphics<5->[width=\textwidth]{nodes.png}
\end{figure}
\end{frame}

\subsection{Tutoriumsaufgabe}
\begin{frame}{Tutoriumsaufgabe}
	siehe pdf
\end{frame}

\subsection{doppelt verkettet}
\begin{frame}[fragile]{Doppelt verkettete Listen}
Jemand eine Idee, wie das realisierbar ist? \pause
\begin{exampleblock}<2->{Beispiel}
\begin{lstlisting}[basicstyle=\scriptsize]
class DNode {
	Object content;
	DNode next; 
	DNode prev;
	
	DNode(content, DNode next, DNode prev) {
		this.content = content;
		this.next = next; 
		this.prev = prev;
	}
}
\end{lstlisting}
\end{exampleblock} \pause

\end{frame}



\appendix
\beginbackup

%\begin{frame}[allowframebreaks]{References}
%	\printbibliography
%\end{frame}

\backupend

\end{document}
