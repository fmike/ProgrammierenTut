%% LaTeX-Beamer template for KIT design
%% by Erik Burger, Christian Hammer
%% title picture by Klaus Krogmann
%%
%% version 2.1
%%
%% mostly compatible to KIT corporate design v2.0
%% http://intranet.kit.edu/gestaltungsrichtlinien.php
%%
%% Problems, bugs and comments to
%% burger@kit.edu

\documentclass[18pt]{beamer}

%% SLIDE FORMAT

% use 'beamerthemekit' for standard 4:3 ratio
% for widescreen slides (16:9), use 'beamerthemekitwide'

\usepackage{templates/beamerthemekit}
% \usepackage{templates/beamerthemekitwide}

%% TITLE PICTURE

% if a custom picture is to be used on the title page, copy it into the 'logos'
% directory, in the line below, replace 'mypicture' with the 
% filename (without extension) and uncomment the following line
% (picture proportions: 63 : 20 for standard, 169 : 40 for wide
% *.eps format if you use latex+dvips+ps2pdf, 
% *.jpg/*.png/*.pdf if you use pdflatex)

%\titleimage{mypicture}

%% TITLE LOGO

% for a custom logo on the front page, copy your file into the 'logos'
% directory, insert the filename in the line below and uncomment it

%\titlelogo{mylogo}

% (*.eps format if you use latex+dvips+ps2pdf,
% *.jpg/*.png/*.pdf if you use pdflatex)

%% TikZ INTEGRATION

% use these packages for PCM symbols and UML classes
% \usepackage{templates/tikzkit}
% \usepackage{templates/tikzuml}

% the presentation starts here
\usepackage{graphicx}
\usepackage{listings}
\usepackage{color}
\usepackage{textcomp}
\definecolor{listinggray}{gray}{0.9}
\definecolor{lbcolor}{rgb}{0.9,0.9,0.9}
\lstset{
	language=Java,
	backgroundcolor=\color{lbcolor},
	tabsize=4,
	rulecolor=,
        basicstyle=\footnotesize,
        aboveskip=5pt,
        upquote=true,
        columns=fixed,
        showstringspaces=false,
        extendedchars=true,
        breaklines=true,
        frame=single,
        showtabs=false,
        showspaces=false,
        showstringspaces=false,
        identifierstyle=\ttfamily,
        keywordstyle=\color[rgb]{0,0,1},
        commentstyle=\color[rgb]{0.133,0.545,0.133},
        stringstyle=\color[rgb]{0.627,0.126,0.941},
}
\usepackage[utf8]{inputenc}
\usepackage{stmaryrd}

\title[Prog Tut Nr. 8]{Tutorium Programmieren}
\subtitle{Tut Nr.8: FileReader, String API}
\author{Michael Friedrich}
\date{17. / 19.12.2013}

\institute{Institut f\"ur theoretische Informatik}
% Bibliography

\usepackage[citestyle=authoryear,bibstyle=numeric,hyperref,backend=biber]{biblatex}
\addbibresource{templates/example.bib}
\bibhang1em

\begin{document}

% change the following line to "ngerman" for German style date and logos
\selectlanguage{ngerman}

%title page
\begin{frame}
	\titlepage
\end{frame}

%table of contents
\begin{frame}{Outline/Gliederung}
	\tableofcontents
\end{frame}

\section{Blatt3}
\begin{frame}{Besprechung Blatt 3}\pause
\begin{itemize}
	\item nur geforderte Dateien hochladen! \pause
	\item achtet auf die Aufgabenstellung! \pause
	\begin{itemize}
		\item Modellierung! $\Rightarrow$ Methoden, nicht alles in die main \pause
		\item Ausgabe! $\Rightarrow$ Leerzeichen können euch in der Abschlussaufgabe teuer kosten\pause
		\item Aufgabe lesen! zB args $\Rightarrow$ bei Unklarheit fragen!
	\end{itemize}
\end{itemize}
\end{frame}

\begin{frame}[fragile]{I made a Boo Boo}\pause
\begin{itemize}
	\item Fehler meinerseits...Kein Abzug, nur Anmerkung in der Korrektur. Aber in Zukunft Fehler!\pause
\begin{itemize}
	\item Exceptions unbedingt zu vermeiden!
\end{itemize}
\end{itemize}
\begin{block}{Lücke füllen und damit implizit löschen}\pause
\begin{lstlisting}
	  for(int j = indexFound; j < elements.length; j++) {
			elements[j] = elements[j+1];
		} 
			lastElem--;
				}
\end{lstlisting}
Führt zu einer Exception! Ähnlich auch in Aufgabenteil C \newline \pause
$\Rightarrow$ Entweder: Nur bis zu (elements.length - 1) gehen und letztes Element außerhalb der for Schleife manipulieren \newline \pause
$\Rightarrow$ ODER : Indizierung anders wählen: elements[j-1] = elements[j] im for-Rumpf
\end{block}
\end{frame}

\section{Blatt4}
\begin{frame}{Hinweise Blatt 4}\pause
\begin{itemize}
\item Kapselung! Jetz Pflicht...Wiederholungsbedarf?\pause
\item "`Faustregeln"' \pause
\begin{itemize}
	\item Attribute private  \pause
	\item Methoden meistens public (außer Hilfsmethoden) \pause
	\item Konstruktor \color[rgb]{1,0,0}{IMMER} public  \pause
\end{itemize} \pause
\item Vergesst nicht, Checkstyle upzudaten \pause
\item Listen erstellen wie letztes Tut 
\end{itemize}
	
\end{frame}

\section{FileReader}
\begin{frame}[fragile]{FileReader}
\begin{columns}[c]
\column[c]{5cm}
\begin{lstlisting}[basicstyle=\tiny]

	public static void main(String[] args) {
	if (args.length != 1) {
		System.out.println(USAGE);
		System.exit(1);
	}
	FileReader in = null;
	try {
		in = new FileReader(args[0]);
	} catch (FileNotFoundException e) {
		System.out.println(ERROR_MESSAGE);
		System.exit(1);
	}
		BufferedReader reader = new BufferedReader(in);
	try {
		String line = reader.readLine();
		while (line != null) {
		// TODO: process line here
		line = reader.readLine();
		}
	} catch (IOException e) {
		System.out.println(ERROR_MESSAGE);
		System.exit(1);
		}
	}
\end{lstlisting}
		\column{5cm}
		\begin{itemize}
		\item FileReader öffnet die Datei \pause
		\item BufferedReader nimmt den Datenstrom aus der Datei entgegen \pause \newline
		$\shortrightarrow$ Datei lässt sich Zeilenweise auslesen \pause
		\item Fehlerbehandlung muss hier geschehen, lernen wir aber erst nächstes Jahr \pause \newline
		$\shortrightarrow$ Ihr könnt das nickend annehmen und benutzen \pause \newline
		$\Rightarrow$ TODO euer Angriffspunkt 
		\end{itemize}
	\end{columns}
\end{frame}

\section{String API}
\begin{frame}[fragile]{String API}
Benötigte Theorie: \textbf{REGular EXpressions} (siehe GBI) hier erstmal Trennzeichen, wie z.Bsp.     . (Punkt) \pause
\begin{block}{Werkzeugkasten, um Strings zu manipulieren}\pause
\begin{itemize}
	\item \lstinline{String[] split(String regex)} Trennt den String am Trennzeichen und liefert ein Array zurück\pause
	\begin{lstlisting}[basicstyle=\scriptsize]
	"boo:and:foo".split(":"); // Ergebnis: {" bo o ","and "," f o o "} 
  "boo:and:foo".split("o"); // Ergebnis: {" b ",""," : and : f "} 
	\end{lstlisting}  \pause
	\item \lstinline{boolean matches(String regex)} Prüft, ob ein String dem regulären Ausdruck entspricht \pause
	\begin{lstlisting}[basicstyle=\scriptsize]
	"A1".matches([A-Z][0-9]) //true 
	"a8".matches([A-Z][0-9]) //false 
	\end{lstlisting} \pause
	\item \lstinline{String toLowerCase()} Macht jedes Zeichen eines String klein\pause
	\begin{itemize}
		\item analog: \lstinline{String toUpperCase()} \pause
	\end{itemize}
\end{itemize}
\end{block}
\end{frame}


\begin{frame}[fragile]{String API}
\begin{block}{Werkzeugkasten, um Strings zu manipulieren} \pause
\begin{itemize}
\item \lstinline{String trim()} Entfernt Whitespaces am Anfang und am Ende des Strings \pause
\item \lstinline{int compareTo} Lexikographischer Vergleich, d.h. alphabetisch \pause
\end{itemize}	 \pause
\begin{lstlisting}[basicstyle=\footnotesize]
	"Adam".compareTo("Adam"); / / Ergebnis == 0 
	"Adam".compareTo("Eva"); / / Ergebnis < 0 
	"Eva".compareTo("Adam"); / / Ergebnis > 0 
	"Adam".compareTo("adam"); / / Ergebnis < 0 
	"Eva".compareTo("adam"); / / Ergebnis < 0 
\end{lstlisting} 
\end{block}
\end{frame}

\section{Quiz}
\begin{frame}[fragile]{JahresAbschlussQuiz}
\begin{itemize}
	\item siehe pdf \newline 
  \vspace{1cm}
  \item Oder: Habt ihr noch WehWehchen, die wir uns anschauen sollen? \newline
		\footnotesize Zum Beispiel: Listen, Sichtbarkeiten, sonstige Anliegen...?
\end{itemize}

\end{frame}



\appendix
\beginbackup

%\begin{frame}[allowframebreaks]{References}
%	\printbibliography
%\end{frame}

\backupend

\end{document}
